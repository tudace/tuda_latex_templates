\documentclass[
	color=3c,
%	logofile=example-image, %Falls die Logo Dateien nicht vorliegen
	]{tudaleaflet}

\usepackage[ngerman]{babel}

\usepackage{blindtext}

%Literaturverweise
\usepackage{biblatex}
\bibliography{DEMO-TUDaBibliography}

%Formatierungen für Beispiele in diesem Dokument. Im Allgemeinen nicht notwendig!
\let\file\texttt
\let\code\texttt
\let\pck\textsf
\let\cls\textsf


\begin{document}
	


%%TODO sponsors
%\renewcommand*{\raggedtitle}{\raggedleft}


\title{TUDaLeaflet}
%\titleimage{\includegraphics[width=\width,height=\height]{example-image}}
\subtitle{Leaflets using TUDa-CI}
\addTitleBox{arg1}
\maketitle


\section{Grundlegende Funktionsweise}
\cls{tudaleaflet} basiert auf der \cls{leaflet}-Klasse. Die Verwendung ähnelt der einer einfachen ArtikelKlasse lediglich die Seitemumbrüche sind auf ein Faltblatt ausgelegt. Es werden üblicherweise 6 Teil-Seiten erzeugt. Für weitere Informationen zur \cls{leaflet}-Klasse, kann \cite{leaflet} hilfreich sein. Optionen, die nicht durch \cls{tudaleaflet} definiert wurden, werden ggf. an die Basisklasse weitergereicht.

Diese ist keine \KOMAScript-Klasse. Dennoch wurde durch \pck{scrextend} einige Mechanismen zur besseren Kompatibilität eingebaut, allerdings sind nicht alle Mechanismen verfügbar \cite[vgl.][]{scrguide}.

\section{Die Titelseite}
Die Titelseite wird analog zu den übrigen TUDa-CI-Klassen erzeugt. Als Makros stehen hierbei die folgenden zur Verfügung:
\begin{description}
	\item[title] Titel
	\item[subtitle] Untertitel, wird im Block zwischen Titel und Titelbild platziert.
	\item[author] Autor, wird ggf. unter dem Untertitel platziert.
	\item[date] Datum, wiederum unter dem ggf. platzierten Autor
	\item[addTitleBox] Institutsboxen.
	Die Institutsboxen werden mit vorgegebenem Abstand unter dem Logo platziert. Hier kann Text erscheinen oder auch ein Institutslogo. Der Hintergrund ist weiß.
	
	Um die Institutsboxen für Logos zu verwenden, liefert \cls{tudaleaflet} analog zu \cls{tudapub} das Makro \code{\addTitleBoxLogo}. Als Argument akzeptiert es einen Bilddateipfad.
	\begin{verbatim}
	\addTitleBoxLogo{example-image}
	\addTitleBoxLogo*{
	  \includegraphics
	    [width=\linewidth]
	    {example-image}
	}
	\end{verbatim}
	\item[titleimage] Titelbild. Das Makro \code{titleimage} akzeptiert beliebigen Inhalt. Dieser wird bündig mit der oberen Ecke im Hauptteil der Titelseite platziert.
	Üblicherweise wird dieses Makro zur Platzierung einer Grafik genutzt:
	\begin{verbatim}
	\titleimage{
	  \includegraphics
	  [width=\width]
	  {example-image}
	}
	\end{verbatim}
	Die Makros \verb+\height+ und \verb+\width+ sind hier lokal so definiert, dass es damit möglich ist, die Fläche zu füllen.
\end{description}

\subsection{Optionen zur Anpassung der Titelseite}
\begin{description}
	\item[colorback] aktiviert die Hintergrundfarbe der \verb+\titleimage+-Fläche. Voreingestellt ist \verb+color+""\verb+back=true+
	\item[colorbacktitle] Analog für den Titelblock, Voreingestellt ist hier \verb+colorbacktitle=false+
	\item[colorbacksubtitle] Diese Option funktioniert nur für \verb+colorbacktitle=true+. Damit ist es möglich, dass auch der Streifen zwischen Titelblock und Bildfläche in der Farbe der Identitätsleiste eingefärbt wird. Falls \verb+colorbacktitle=false+ erzeugt dies eine Fehlermeldung. Voreingestellt ist \verb+colorbacksubtitle=false+
	\item[logofile] Alternatives Hauptlogo, z.\,B. falls das TUDa-Logo nicht vorliegt.
\end{description}

\printbibliography


\end{document}