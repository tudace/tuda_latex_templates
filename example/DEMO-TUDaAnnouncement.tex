\documentclass[
	paper=a4,
	ngerman,
	accentcolor=9c,
	footer=true,
	type=announcement,
	fontsize=12pt
%	logofile=example-image, %Falls die Logo Dateien nicht vorliegen
	]{tudaposter}

\usepackage[english, main=ngerman]{babel}
\usepackage[babel]{csquotes}


%Formatierungen für Beispiele in diesem Dokument. Im Allgemeinen nicht notwendig!
\let\file\texttt
\let\code\texttt

\begin{document}

\title{\LaTeX{} im Corporate Design der TU~Darmstadt}
\subtitle{Aushänge mit tudaposter}

\footerqrcode{https://peitex.de}
\footer{Inhalt der Fußzeile}%Falls aktiviert


\maketitle

\section*{Grundlengende Informationen}
Die Dokumentenklasse tudaposter dient der Erstellung von Aushängen und Nicht"=wissenschaftlichen Plakaten im Stil der TU-Darmstadt. Sie ist Teil des TUDa\TeX-Bundles.

Für dieses Dokument gelten die in DEMO-TUDaPoster beschriebenen Mechanismen. Zusätzlich wird jedoch ein Modus aktiviert, der entsprechend des Corporate Design Handbuchs für Aushänge verwendet werden soll.

Die Option \code{type=announcement} ändert dementsprechend den Texteinzug, dass dieser identisch zum Einzug im Titelblock ist. Darüber hinaus wird hier das Logo immer in der Kopfzeile platziert und der Untertitel in Fettschrift gesetzt.

\end{document}
