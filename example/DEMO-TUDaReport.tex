\documentclass[
	ngerman,
	accentcolor=9c,% Farbe für Hervorhebungen auf Basis der Deklarationen in den
	type=intern,
	marginpar=false
	]{tudapub}

\usepackage[english, main=ngerman]{babel}
\usepackage[babel]{csquotes}

%Formatierungen für Beispiele in diesem Dokument. Im Allgemeinen nicht notwendig!
\let\file\texttt
\let\code\texttt
\let\pck\textsf
\let\cls\textsf

\begin{document}


\title{TUDaReport -- Minimales Template für TUDa-CI}
\author{Marei Peischl}
%\date{} % Ohne Angabe wird automatisch das heutige Datum eingefügt

\maketitle

\tableofcontents

\section{Über diese Datei}
Die Datei \file{DEMO-TUDaReport.tex} beziehungsweise ihre Ausgabe \file{DEMO-TUDaReport.pdf} ist ein minimales Template für die Verwendung Der Dokumentensklasse \file{tudapub.cls}.


Für Erläuterungen zur Funktionsweise und den Möglichkeiten zur Änderung sei auf DEMO-TUDaPub.pdf verwiesen.
\end{document}
